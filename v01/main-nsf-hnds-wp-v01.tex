\documentclass[11pt]{article}
\usepackage{etoolbox}
\newtoggle{withinstructions}
%\toggletrue{withinstructions} % comment this if you want to hide the comments

\usepackage{enumitem}
\setlist[itemize]{leftmargin=*}
\usepackage{ifxetex}
\ifxetex
\usepackage{fontspec} % this package requires using xelatex or lualatex engine instead of pdflatex. In TexShop, change the preferences->Engine->Latex box
% \setmainfont[Ligatures=TeX]{Palatino Linotype} % uses actual Palatino under the font spec package
% \setsansfont[Ligatures=TeX]{Palatino Linotype} % uses actual Palatino under the font spec package
% \setmonofont[Ligatures=TeX]{Palatino Linotype} % makes \tt commands use Palatino under the font spec package
% \fontspec[Ligatures=TeX]{Palatino Linotype} % uses actual Palatino under the font spec package

%    --- this was helpful, arial -------
\setmainfont[Ligatures=TeX]{Arial} % uses actual Palatino under the font spec package
\setsansfont[Ligatures=TeX]{Arial} % uses actual Palatino under the font spec package
\setmonofont[Ligatures=TeX]{Arial} % makes \tt commands use Palatino under the font spec package
\fontspec[Ligatures=TeX]{Arial} % uses actual Palatino under the font spec package

%  try now palatino:  this did not work as well.
%\setmainfont[Ligatures=TeX]{Palatino Linotype} % uses actual Palatino under the font spec package
%\setsansfont[Ligatures=TeX]{Palatino Linotype} % uses actual Palatino under the font spec package
%\setmonofont[Ligatures=TeX]{Palatino Linotype} % makes \tt commands use Palatino under the font spec package
%\fontspec[Ligatures=TeX]{Palatino Linotype} % uses actual Palatino under the font spec package


\else
\usepackage{mathptmx}% http://ctan.org/pkg/mathptmx for Times (but not really Times New Roman)
\fi

\usepackage{scrextend}
\KOMAoption{fontsize}{10.20pt}
%\KOMAoption{fontsize}{10.04pt}       %<----- this is for real; cjk
\usepackage{hyperref}  % comment out to remove hyperlinks from PDF file produced
\usepackage{microtype}
\usepackage[left=1in,top=1in,right=1in,bottom=1in]{geometry}
%\usepackage[left=.52in,top=.52in,right=.52in,bottom=.52in]{geometry}
\usepackage{amsmath,amsthm,amssymb}
\usepackage{graphicx}
\usepackage{enumitem}
\usepackage{titlesec}
%\titlespacing{\section}{0pt}{*0}{0pt} %For no space around section headings
\titlespacing\section{0pt}{6pt plus 4pt minus 2pt}{0pt plus 2pt minus 2pt}
%\titlespacing\subsection{0pt}{6pt plus 4pt minus 2pt}{0pt plus 2pt minus 2pt}
%\titlespacing\subsubsection{0pt}{6pt plus 4pt minus 2pt}{0pt plus 2pt minus 2pt}
%\titlespacing\subsection{0pt}{2pt plus 4pt minus 2pt}{1em plus 2pt minus 2pt}
\titlespacing\subsection{0pt}{2pt plus 4pt minus 2pt}{0pt plus 2pt minus 2pt}
\titlespacing\subsubsection{0pt}{0pt plus 4pt minus 2pt}{1em plus 2pt minus 2pt}
\titlespacing\paragraph{0pt}{0pt plus 4pt minus 2pt}{1em plus 2pt minus 2pt}
%\titleformat{\subsection}[runin]{\normalfont\bfseries\large}{}{0 pt}{}[.]
\titleformat{\subsection}{\normalfont\bfseries\large}{\thesubsection}{5 pt}{}
\titleformat{\subsubsection}[runin]{\normalfont\bfseries}{}{0 pt}{}[.]
\titleformat{\paragraph}[runin]{\normalfont\itshape}{}{0 pt}{}[.]
% number aims
\newcounter{aimno}
\newcommand{\aim}[1]{\refstepcounter{aimno}\label{#1}}


\usepackage{wrapfig}
%\usepackage{subfig}
\usepackage[labelformat=simple]{subcaption}
\renewcommand\thesubfigure{(\alph{subfigure})}
\usepackage{cite}
\usepackage{color}
%\usepackage{apacite}
\usepackage{mdwlist}
\usepackage{soul}
\usepackage{multirow}
\usepackage{csquotes}
\usepackage{url}
% I commented these out because they interact badly with the font spec package, which is needed to get the required Palatino at 11 Adobe points.
% If you need them back, tell me so we can figure out a work-around.  - SE
%\renewcommand{\rmdefault}{phv} % Arial
%\renewcommand{\sfdefault}{phv} % Arial
\usepackage[sort&compress,numbers]{natbib}
\setlength{\bibsep}{0.0pt}
\usepackage{framed}

\usepackage{booktabs}% http://ctan.org/pkg/booktabs
\newcommand{\tabitem}{~~\llap{\textbullet}~~}
\usepackage{tabularx}
\usepackage[font=small,labelfont=bf]{caption}
\usepackage[colorinlistoftodos]{todonotes}
\newcommand{\TODO}[1]{\todo[inline,caption={},color=red!20,size=\small]{#1}}
\newcommand{\NOTE}[1]{\todo[inline,caption={},color=blue!20,size=\small]{#1}}
\def\comment#1{{\color{red}[#1]}}
%%%%% UNCOMMENT these to disable boxes
%\renewcommand{\TODO}[1]{}
%\renewcommand{\NOTE}[1]{}

\renewcommand\labelitemi{\rule[0.12em]{0.4em}{0.4em}}
\renewcommand\labelitemii{\normalfont\bfseries \rule[0.15em]{0.3em}{0.3em}}

\usepackage[mode=text]{siunitx}
% Setup inline list environment for inline roman numeral lists e.g., (i) ... (ii) ...
\newlist{inline}{enumerate*}{1}
\setlist[inline]{before=\unskip{: }, itemjoin={{; }}, itemjoin*={{; and }}, label=\textit{(\roman*)}}
\setlist{nosep}
\usepackage{xspace}
\newcommand{\scidis}{SciDIS\xspace}
%\usepackage{helvet}
%\usepackage{palatinox}
\thispagestyle{empty}
\pagestyle{empty}

%\begin{document}


%%\tableofcontents
%%\newpage


\baselineskip = 1.12\normalbaselineskip    %     <------------  comment this out, when for real;   cjk
%\setlength{\parskip}{1pt}
%
%\thispagestyle{empty}










%
%
%
%% !TEX root = ./main-real-proj-descr.tex
%%%
%%%
%%%
%
%\documentclass[11pt]{article}
%
%% Set the page space.
%\usepackage[top=1.in, bottom=1in, left=1.1in, right=1.1in]{geometry}
%
%
%
%
%
%
%
%
%
%
%
%
%\usepackage{url}
%\usepackage{verbatim}
%\usepackage{xspace}
%\usepackage{amsmath}
%\usepackage{amssymb}
%\usepackage{color}
%\usepackage{graphicx}
%\usepackage[square,sort&compress,numbers]{natbib}
%%\usepackage{graphicx,wrapfig,lipsum}
%\usepackage{wrapfig,lipsum}
%
%
%
%% cjk
%\usepackage{ulem}
%\normalem
%\usepackage{pifont}
%\usepackage{xtab}
%\usepackage{multirow}
%
%\usepackage{latexsym}
%\usepackage{amsfonts}%
%\usepackage{amssymb}%
%
%\usepackage{subfigure}
%\usepackage{float}
%\usepackage{url}
%\usepackage{dsfont}
%
%% Algorithms.
%\usepackage[ruled, vlined]{algorithm2e}
%

% Maksud
%\usepackage{booktabs}
%\usepackage[bookmarks=true]{hyperref}
%\usepackage{bookmark}

\makeatletter
\def\ParHeading{\@ifnextchar[{\@with}{\@without}}
\def\@with[#1]#2{\pdfbookmark[4]{#1}{#2}\vspace{0.1em}\noindent\textbf{#2.}}
\def\@without#1{\pdfbookmark[4]{#1}{#1}\vspace{0.1em}\noindent\textbf{#1.}}
\makeatother

%which also requires:


%\linespread{1.15}

%% Horizontal lines in doc.
\newcommand{\HRule}{\rule{\linewidth}{0.5mm}}


%\newtheorem{theorem}{Theorem}[section]
%\newtheorem{lemma}[theorem]{Lemma}
%\newtheorem{proposition}[theorem]{Proposition}
%\newtheorem{corollary}[theorem]{Corollary}
%
%\theoremstyle{definition}
%\newtheorem{definition}[theorem]{Definition}
%\newtheorem{example}[theorem]{Example}
%\newtheorem{remark}[theorem]{Remark}
%\newtheorem{observation}[theorem]{Observation}
%\newtheorem{xca}[theorem]{Exercise}

%% \usepackage{tikz}
%% \usetikzlibrary{arrows,matrix}
%%
%% ------------------------------------------------------------
\graphicspath{{./}{./figures/}}
%% ------------------------------------------------------------
%%
%% Layout:
%\parindent0pt
%\parskip6pt
%%
%% Macros:
\def\indicator{\mathbb{I}}
\def\Exp{\mathbb{E}}
\def\prob{\mathrm{Pr}}
\def\gds{GDS\xspace}
\def\sds{SDS\xspace}
\def\gca{GCA\xspace}
\def\etgds{ET-GDS\xspace}
\def\etsds{ET-SDS\xspace}
\def\etgca{ET-GCA\xspace}
\def\F{\mathbb{F}}
%%\def\Fmap{\mathbf{F}}
\def\Fmap{F}
\def\G{\mathbf{G}}
\def\Per{\mathrm{Per}}
\def\Fix{\mathrm{Fix}}
\def\vset{\mathrm{v}}
\def\eset{\mathrm{e}}
\def\N{\mathbb{N}}
\def\Circle{\mathrm{Circle}}
\def\Path{\mathrm{P}}
\def\Fib#1{\mathrm{Fib}_{#1}}
\def\fib{\mathrm{Fib}}
\def\Luc{\mathrm{Luc}}
\def\supp{\ensuremath{\operatorname{supp}}}

\def\kup{k_{01}}
\def\kdown{k_{10}}
\def\kql{K_{q,l}}
\def\kqsl{K_{q_l}}
\def\kqsls{K_{q_{l^*}}}
\def\kqslm{K_{{q_l},m}}
\def\kqslsm{K_{{q_{l^*}},m}}
\def\l1m{L_{m}}
\def\kqlhat{K_{q_{\hat{l}}}}
\def\kqlstar{K_{q_{l^*}}}

\def\hatz{\hat{0}}
\def\hato{\hat{1}}

\def\Ki{K^i}
\def\Kiq{K^i_{q_i}}
\def\num{\eta}

%% E family.
\def\efam{\mathcal{E}}
%% E family member
\def\efm{E^i_{d_{ext},n}}
\def\efset{E^i_{j,k}}

%%% For 2-column paper.
%\newcommand{\smallPythonSize}{0.19}
%\newcommand{\biggerPythonSize}{0.25}


\def\th{\ensuremath{{}^\text{th}}\xspace}

\def\card#1{\#\bigl(#1\bigr)}
\def\card#1{\bigl|#1\bigr|}
%%
\definecolor{red}{rgb}{1.0,0.0,0.0}
\def\highlight#1{{\color{red}#1}}

% The name of the cyberinfrastructure.
% cyberinfrastructure for network science and engineering.
% Use by typing:  \codeName{}
\newcommand{\codeName}{CINES}

% Page numbers.
\pagestyle{plain}

%%
%% ----------------------------------------------------------------------
%%
\begin{document}
%%
\setlength{\parskip}{1pt}

\thispagestyle{empty}





%%%
%\thanks{$^*$ Corresponding author: email: TBD;
%  Telephone: +1 540 ???-????; Fax: +1 540 231 2891}
%%%
%\keywords{Keywords: Twitter, social networks, follower graph, network generation, network evolution, network prediction}
%%
%\begin{abstract}
%Follower graphs.
%\end{abstract}
%%
%\maketitle
%%

%NSF Call:\\
%Software Infrastructure for Sustained Innovation (SI2:SSE \& SSI)
%Software Elements and Frameworks
%PROGRAM SOLICITATION
%NSF 16-532
%URL: 
%\begin{verbatim}
%http://www.nsf.gov/pubs/2016/nsf16532/nsf16532.pdf
%\end{verbatim}


%% ------------------------------------------------------------
% Generate Outline.

%\tiny
%
%\HRule \\[0.05cm]
%
%\HRule \\[0.05cm]
%
%\small
%
%\tableofcontents
%
%\tiny
%
%\HRule \\[0.05cm]
%
%\HRule \\[0.05cm]
%
%\normalsize



% "pd" means project description.
%% ------------------------------------------------------------
% Wjhat is missing.
%\input{missing.tex}




%% ------------------------------------------------------------
% Project description outline.
 %% ------------------------------------------------------------
\noindent
\textbf{Overview.}
%
\noindent
\textbf{\emph{\underline{Opportunity and Challenge.}}}
Explore-exploit phenomena appear in many practical settings.
Firm innovation~\cite{}



\iffalse
Networks
are ubiquitous and are a part of our common vocabulary. Network
science and engineering that emerged as a formal field over the
last 20 years has seen explosive growth.
Ideas from network science have
played a central role in the formation of companies such as Akamai,
Twitter, Google, Facebook and LinkedIn. 
The concepts have also been used to
address fundamental problems in diverse fields, e.g., epidemiology
and marketing and are now part of most university curricula. Network
science is multi-disciplinary.
%%-- concepts are borrowed from
%% diverse field to advance the science.  
Yet, resources for doing
network science are largely dispersed and stand-alone 
%(in silos of isolated tools), 
of small scale or home-grown for personal use.
Furthermore, many researchers interested in the study of networks are
not computer scientists; also, they seldom have easy access
to computing and data resources.
%%--- this creates a barrier for researchers.  
What is needed is a cyberinfrastructure to bring
together various resources, to provide a unifying ecosystem for
network science that is greater than the sum of its parts. 
%%Similar
%%investments by the NIH and NSF have proven extremely useful in
%%advancing other scientific fields.
\fi


%
\noindent
\textbf{\emph{\underline{Vision.}}}  Our vision is to build a
self-sustaining cyberinfrastructure (\codeName{}) that will be a
\emph{community} resource for network science.  \codeName{}  will
be an \emph{extensible} platform for producers and consumers of
network science data, information, and software. 
%%It will work on an
%%information economy paradigm.  
\codeName{} will have 
({\em i}) a robust and reliable
infrastructure that provides access to networks, software, and computing
as a service that scales under high system load to networks with
$10^{8}$ vertices or more; 
({\em ii}) a
digital library with 100,000+ networks of various kinds that allows
rich services for storing, searching, annotating and browsing; 
({\em iii}) structural methods (e.g., centrality, paths, cuts, etc.) and
dynamical models of various contagion processes;
({\em iv}) new methods to acquire data, to build networks and
augment them using machine learning techniques;
({\em v}) a suite of web-apps that makes it
easier for researchers, educators and analysts to do network science
and engineering; 
({\em vi}) an engine to enable users to create new
workflows by composing available components;  
({\em vii}) a suite of APIs that will allow developers
to add new web-apps and services, based on an app-store model,
and allow access to \codeName{} from third party software; and
({\em viii}) innovative ways to
support reproducibility of results, thereby serving as a powerful
benchmarking and validation resource for journals, conferences and
scientists. 
%and 
%({\em ix}) ways to grow and sustain
%by including an app-store model that facilitates contributions by 
%users and providing incentives to users to contribute data, software
%and educational materials.   
%under the information economy model.

%


\iffalse
%%%%%%%%%
\noindent
\textbf{\emph{\underline{Key Objectives.}}}
We will devise a set of specifications for data and meta-data representations, data formats, APIs,
and input and output data types for applications, which are critical for an interoperable system of
the scale we envision.
We will devise a service oriented architecture (SOA)-based workflow system (using open source
software tools as appropriate) so that disparate 
softwares can be composed by users to perform more complicated, automated tasks.
Services such as visualization, query, and sense-making modules will be developed
to support a broad range of applications.
Several existing tools will be integrated, including the SNAP, NetworkX, and Galib network structural
analysis libraries.
The EDISON system for contagion dynamics simulations will be further developed and integrated.
New tools will be added.
We will use as a starting point the Cyberinfrastructure for Network Science (CINET),
developed in part through NSF funding (Federal Award ID:  1032677).\\
%


\noindent
\textbf{\emph{\underline{Impact.}}}
%The CI will provide a platform in which individuals and teams may contribute software and networks.
Functionality and usability are key to establishing a large
user base, which will in turn motivate the community to contribute networks and software (to increase the
reach and visibility of their research).
All contributions will be attributed to the producer, and metrics on their use will be tallied
so that she can claim credit for materials and their use.
In this way, we will achieve a self-sustaining community resource that can be used for 
research, teaching, and applications.
Furthermore, by provide functionality through both user interfaces (UIs) and APIs, the entire spectrum
of users from novice to expert can make best use of the system.
We expect this cyberinfrastructure to stimulate advances by other researchers, since they can leverage
capabilities supplied by others.\\
%capabilities (e.g., software, datasets, workflows) supplied by others.\\
%Since software and data can be marked private (for some period of time), the cyberinfrastructure will
%enable experimentation with new concepts.\\
%

%%%%%%%
\fi

\noindent
\textbf{Intellectual Merit.} \codeName{} will enable fundamental
changes in the way researchers study and teach complex networks.
The use of state-of-the-art high performance computing (HPC)
resources to synthesize, analyze
and reason about large networks will enable researchers and
educators to study networks in novel ways. \codeName{}  will allow
scientists to address fundamental scientific questions---e.g.,
biologists can use network methods to reason about
genomics data that is now available in large quantities due to fast
and cost effective sequencing and the Microbiome Program.  It will
enable educators to harness HPC technologies to teach
Network Science to students spanning various academic levels,
disciplines and institutions. It will be designed for scalability,
usability, extensibility and continuity. This project will also
advance the fields of digital libraries and cloud computing by
stretching them to address challenges related to Network Science.
Given the multi-disciplinary nature of the field, \codeName{} will
provide a collaborative space for scientists from different disciplines
to interact, leading to important cross fertilization of ideas.


\iffalse
%%%%%%%%%
The workflow system will enable researchers to compose novel processing pipelines through 
combinations of UI- and API-based specifications.
This research will develop new ways to explore analysis results at scale, across a wide range
of application compositions.
New methods to visualize and query big data will be devised, such as increasing the amount of
data that can be plotted.
We anticipate new automated ways to stress test cyberinfrastructures under load and improve reliability.
New modeling techniques, such as integration of domain specific languages, will increase the 
range of applications that can be studied.\\
%%%%%%
\fi

%
\noindent
\textbf{Broader Impact.}

\begin{enumerate}
    \item Virginia Tech's Kids Tech to engage grade school students.
    \item Virginia Tech's Achievable Dream program, aimed at disadvantaged
    high school students in Virginia.
    \item Modeling techniques will be used in two
    statistics courses on data analytics by Professor Xinwei Deng of VT.
    \item A workshop on human behavior modeling will be prepared and 
    given as part of a recurring workshop series presented by 
    advance research computing (ARC) at VT. \
    \item PhD students will participate in these activities.
\end{enumerate}







%\bibliographystyle{siam}
%%\bibliographystyle{abbrvnat}
%%\bibliography{cinet-prop-refs,reliability-refs,msmsort,cms-refs-cinet-proj,sdci10,ed_fox_refs,netsci-refs,netsci-names}  %,refs_education}

%% ------------------------------------------------------------



% Put back in.
%\input{issues}

%\input{sow-options.tex}

\end{document}
